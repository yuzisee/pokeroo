
\newcommand{\xs}{$\spadesuit$}
\newcommand{\xh}{$\heartsuit$}
\newcommand{\xc}{$\clubsuit$}
\newcommand{\xd}{$\diamondsuit$}



\clearpage


\chapter{Poker Hand Evaluation}
\label{sec:HandEvaluation}

The core of the decision engine will contain a mechanism for determining whether a set of cards is stronger than another.
%A hand evaluator is required in order to determine whether a set of cards is stronger than another.
This section presents a data structure for representation of a set of cards, as well as an efficient algorithm for computation of card strength for comparison.


\section{Poker Hands}
\label{sec:PokerHands}


From strongest to weakest the standard poker hands are:

\begin{itemize}
\singlespacing
\item Straight Flush: Five consecutive card values, all with the same suit
\item Four-of-a-kind: Four matching cards
\item Full house: Three-of-a-kind and then a pair
\item Flush: Five cards with the same suit
\item Straight: Five consecutive card values, any suit
\item Three-of-a-kind: Three matching cards
\item Two-pair: Two different pairs
\item One pair: Two matching cards
\end{itemize}
In pair-type poker hands (eg. two-pair, full house, etc.) a higher valued match is stronger than a lower valued match.
In a full house the three-of-a-kind is more important than the pair.
For example, 555QQ is stronger than 444KK, because three fives is stronger than three fours.
Similarly, 99332 is stronger than 7766K, the highest pair, nines, is stronger than the opposing highest pair, sevens.
In non-pair-type poker hands, or when the matches are identical between both players, the values of the remaining cards are used to break ties.
The highest tiebreaking card wins.
Aces are always considered high except when used for the ace-low (5432A) straight, and suits have no associated strength.
%Since all poker hands are five card hands, the poker hands that don't use all five cards (eg. four-of-a-kind, two-pair, etc.) use the remaining cards to break ties.
If it is still a tie, the next highest tiebreaking card is compared next, until all five cards have been used.
In the case that no poker hand is made, all five cards end up being used in this way --- this is commonly referred to as a 'high card' hand.
If a tie cannot be broken with all five cards, neither hand wins.

Holdem poker uses these standard five-card poker hands for strength comparison, however players always have seven available cards from which they will form the best five-card poker hand.


\section{Implementation}
\label{sec:HandEvaluationImplementation}

Hand evaluation exists at two levels.
Firstly, it must be possible to compute which seven-card hand would win among a set of seven-card hands.
This requires an implicit computation of the best five-card poker hand that can be formed within each seven-card hand.
Secondly, it will be necessary to evaluate the strength of a set of incomplete hands, that is, those with fewer than seven cards.
This is then determined by the distribution of potential seven-card hands that can be formed.

%In order to generate probability distributions to describe $Pr_\mathrm{*} \left( f_\mathrm{bet} | H \right)$ it is necessary to implement a ... evaluator.

%The ... evaluator uses Holdem-specific assumptions to improve efficiency.

%For the purposes of implementation it is safe to assume that strengths will only be compared between seven-card hands.


\subsection{Data Structure}
\label{sec:DataStructure}

The core poker hand evaluator takes a seven cards as input, and produces a comparable card 'strength' object as output.
The strength of any seven cards can be represented with two values: the poker hand made and the tiebreaking cards.

By storing a set of cards as bitfields, the strength information can be computed efficiently.
A set of cards is stored as an array of four 16-bit integers (the 'cardset' array), and a single 32-bit integer (the 'valueset' integer).

Each of the four 16-bit 'cardset' integers represents an individual suit.
Each bit in each 'cardset' integer determines whether that card is contained in the set.
Storing aces as both bit 0 and bit 13 improves efficiency of the strength calculation stage.
\textbf{Table~\ref{tab:CardsetBitfield}} provides an example of the 'cardset' bitfield.

The 'valueset' integer determines how many times each card is contained in the set.
Every two bits are used to store a number between 0 and 3.
Although this representation is unable to store hands with four-of-a-kind, it will be shown that the the valueset variable is not needed in four-of-a-kind situations.
\textbf{Table~\ref{tab:ValuesetInteger}} provides an example of the 'cardset' bitfield.

\begin{table}[htb]
\captionsetup{position=top}
\caption[Cardset Bitfield]{Each integer in the cardset array represents a separate suit.
In this example, index 0, 1, 2, and 3 represent space, hearts, clubs, and diamonds, respectively.
The set of cards represented by (3080, 8, 8193, 14) is Q\xs{} J\xs{} 3\xs{} 4\xh{} A\xc{} 4\xd{} 3\xd{} 2\xd{}.
Bits 14 and higher are unused.}
\begin{small}
\begin{center}
\begin{tabular}{|r|c|c|c|c|c|c|c|c|c|c|c|c|c|c|c|c|r|}
\hline
Bit                 &  \ordinalnum{13} & \ordinalnum{12} & \ordinalnum{11} & \ordinalnum{10} & \ordinalnum{9} & \ordinalnum{8} & \ordinalnum{7} & \ordinalnum{6} & \ordinalnum{5} & \ordinalnum{4} & \ordinalnum{3} & \ordinalnum{2} & \ordinalnum{1} & \ordinalnum{0} &                        \\ \hline
Card                &                A &               K & Q               & J               & 10             &              9 &              8 & 7              &              6 &  5             &              4 &              3 & 2              & A              &                         \\ \hline
                    &                  &                 &                 &                 &   &   &   &   &   &   &   &   &                         &                     & Integer                       \\ \hline
\texttt{cardset[0]} &     0            &        0        & 1               &  1              &  0 & 0  &  0 &  0 &  0 &0   &  1  & 0    & 0    & 0     & 3080   \\
\texttt{cardset[1]} &     0            &    0            & 0               & 0               &  0 & 0  & 0  &  0 &  0 & 0  &  1  & 0    & 0    &  0    & 8   \\
\texttt{cardset[2]} &     1            &    0            &  0              &  0              &  0 & 0  &  0 & 0  &  0 &  0 & 0   &  0   &  0   &  1    & 8193           \\
\texttt{cardset[3]} &     0            &     0           &  0              &  0              &   0 &   0&  0 &  0 &   0& 0  & 1  &  1   &  1   & 0     & 14      \\
\hline
\end{tabular}
\label{tab:CardsetBitfield}
\end{center}
\end{small}
\end{table}

\begin{table}[htb]
\captionsetup{position=top}
\caption[Valueset field]{The 'valueset' integer corresponding to the cardset bitfield example of \textbf{Table~\ref{tab:CardsetBitfield}}.
Every pair of bits denotes how many occurances of that value exist in the set of cards.
In this case, there is one A, one Q, one J, three fours, a three, and a two.
The actual integer value is 72351957.}
\begin{small}
\begin{center}
\begin{tabular}{|r|c|c|c|c|c|c|c|c|c|c|c|c|c|c|c|c|}
\hline
Bits    &  27-26 & 25-24 & 23-22 & 21-20 & 19-18 & 17-16 & 15-14 & 13-12 & 11-10 & 9-8 & 7-6 & 5-4 & 3-2 & 1-0                       \\ \hline
Card    &    A    &     K & Q     & J     & 10    &     9 &     8 & 7     &     6 &  5  &   4 &   3 & 2   & A                          \\ \hline
Value  &    1    &0      & 1     &  1    &  0    & 0     &  0    &  0    &  0    &0    &  3  & 1   & 1   & 1     \\
\hline
\end{tabular}
\label{tab:ValuesetInteger}
\end{center}
\end{small}
\end{table}

The 'cardset' bitfield is used to efficiently determine whether a certain card is contained in the hand.
The 'valueset' bitfield can efficiently determine how many cards of a certain value are contained in the hand.
Furthermore, the ordinality of the 'valueset' bitfield is very similar to what is required for tiebreaking.

%For the purposes of strength calculation, 

\subsection{Strength Calculation}
\label{sec:StrengthCalculation}

The first step of the strength calculation involves detecting what poker hand has been made.
Detecting straights, straight flushes, and four-of-a-kind takes advantage of the 'cardset' representation.
Detecting pairs, triples and flushes is performed incrementally, as each card is added to the set.
The strength of a hand is stored in one byte as outlined in \textbf{Table~\ref{tab:StrengthByte}}.

Straight flushes can be detected by shifting and ANDing each of the 'cardset' bitfields.
Straights can be detected by looking for straight flushes within a combined bitfield that is the bitwise OR of the four 'cardset' bitfields.
To detect flushes, any time a card is added to the set, a corresponding counter is incremented for that suit.
Any counter exceeding five indicates that the poker hand contains at least a flush.
%by counting the number of 1s in each cardset bitfield (excluding the Ace at bit 0 so that it is not double-counted).
%Card matches (pairs, two-pairs, three-of-a-kind, full house, four-of-a-kind) are also detected incrementally.
A list of the top two pairs is always stored, and the list is updated whenever another pair occurs.
If a card with a value that has already paired is added to a set of cards, the hand is promoted to a three-of-a-kind, or to a full house if another pair already exists.
The same applies to four-of-a-kind.


\subsection{Tiebreaking}
\label{sec:Tiebreaking}
Two hands that have the exact same strength (eg. two-pair vs. the same two-pair) require tiebreaking.
For a set of five cards where either a flush or no poker hand is made, the ordinality of the 'valueset' bitfield directly determines the winner.
For a set of seven cards, the two cards with the least value need to be subtracted from the 'valueset' bitfield.
This operation is sufficient for all hands up to and including three-of-a-kind: The smallest two cards that aren't used in a pair/triplet are subtracted from the 'valueset' bitfield, and the remaining value represents the tiebreaker.
The tiebreak value of a straight or straight flush only depends on one of the cards in the straight, and the tiebreak value of a four-of-a-kind depends only on the highest remaining card.
Lastly, during a full-house, the tiebreak variable needs only to represent the contained three-of-a-kind and pair, with more significance placed on the three-of-a-kind.

Organizing the strength and tiebreak fields in this way maps each poker hand strength to a set of integers, while preserving ordinality.
This representation allows for efficient sorting and searching of arbitrary groups of seven-card hands, leading to a mechanism for evaluating incomplete hands by iterating over all possible complete hands that can be obtained.
The combinations of all strength and tiebreak pairs are summarized in \textbf{Table~\ref{tab:StrengthByte}}.


\begin{table}[htb]
\captionsetup{position=top}
\caption[Hand Strength Byte]{The strength of a poker hand can be represented with a single unsigned byte.
%The ordinality of each number reflects the relative strength of its corresponding poker hand.
There are 13 possible pairs, three-of-a-kinds, and four-of-a-kinds: A,K,...,3,2.
There are $\binom{13}{2}$ possible two-pairs: AK,AQ,AJ,...,54,53,52,43,42,32.}
\begin{small}
\begin{center}
\begin{tabular}{|r|c|l|}
\hline
Poker Hand      & Strength & Tiebreaker                                   \\ \hline
Straight Flush  & 122      & High-card of straight                        \\%& Ace is high except in 5432A straight \\
Four-of-a-kind  & 121-109  & Highest non-four-of-a-kind card              \\
Full house      & 108      & Three-of-a-kind, then best pair              \\
Flush           & 107      & Valueset, lowest two cards removed           \\
Straight        & 106      & High-card of straight                        \\
Three-of-a-kind & 105-93   & Valueset, lowest two unpaired cards removed  \\
Two-pair        & 92-15    & Valueset, lowest two unpaired cards removed  \\
One Pair        & 2-14     & Valueset, lowest two unpaired cards removed  \\
Nothing         & 1        & Valueset, lowest two cards removed           \\
\hline
\end{tabular}
\label{tab:StrengthByte}
\end{center}
\end{small}
\end{table}


\subsection{Iteration}
\label{sec:Iteration}


\clearpage
