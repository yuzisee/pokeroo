
\clearpage


\chapter{Implementation}
\label{sec:Implementation}

In order to generate probability distributions to describe $Pr_\mathrm{*} \left( f_\mathrm{bet} | H \right)$ it is necessary to implement a hand evaluator.

\section{Hand Evaluator}
\label{sec:HandEvaluator}

%The hand evaluator uses Holdem-specific assumptions to improve efficiency.
The core hand evaluator takes a seven card hand as input, and produces a comparable card 'strength' object as output.
The strength of any seven card hand can be represented with two values: the poker hand made and the tiebreaking cards.

By storing a set of cards as bitfields, the strength information can be computed efficiently.
A set of cards is stored as an array of four 16-bit integers (the 'cardset' array), and a single 32-bit integer (the 'valueset' integer).

\subsection{Hand Storage}
\label{sec:HandStorage}

Each of the four 16-bit 'cardset' integers represents an individual suit.
Each bit in each 'cardset' integer determines whether that card is contained in the hand.
Storing aces as both bit 0 and bit 13 improves efficiency of the strength calculation stage.
\textbf{Table~\ref{tab:CardsetBitfield}} provides an example of the 'cardset' bitfield.

The 'valueset' integer determines how many times each card is contained in the hand.
Every two bits are used to store a number between 0 and 3.
Although this representation is unable to store hands with four of a kind, the valueset integer is not needed anyways when computing four of a kind hand strengths.
\textbf{Table~\ref{tab:ValuesetInteger}} provides an example of the 'cardset' bitfield.

\begin{table}[htb]
\captionsetup{position=top}
\caption[Cardset Bitfield]{The 'cardset' bitfield for storing poker hands. Each integer in the cardset array represents a separate suit.
In this example, index 0, 1, 2, and 3 represent space, hearts, clubs, and diamonds, respectively.
The hand represented by (3080, 8, 8193, 14) is QJ3 of spades, 4 of hearts, A of clubs, and 432 of diamonds.
Bits 14 and higher are unused.}
\begin{small}
\begin{center}
\begin{tabular}{|r|c|c|c|c|c|c|c|c|c|c|c|c|c|c|c|c|r|}
\hline
Bit                 &  \ordinalnum{13} & \ordinalnum{12} & \ordinalnum{11} & \ordinalnum{10} & \ordinalnum{9} & \ordinalnum{8} & \ordinalnum{7} & \ordinalnum{6} & \ordinalnum{5} & \ordinalnum{4} & \ordinalnum{3} & \ordinalnum{2} & \ordinalnum{1} & \ordinalnum{0} &                        \\ \hline
Card                &                A &               K & Q               & J               & 10             &              9 &              8 & 7              &              6 &  5             &              4 &              3 & 2              & A              &                         \\ \hline
                    &                  &                 &                 &                 &   &   &   &   &   &   &   &   &                         &                     & Integer                       \\ \hline
\texttt{cardset[0]} &     0            &        0        & 1               &  1              &  0 & 0  &  0 &  0 &  0 &0   &  1  & 0    & 0    & 0     & 3080   \\
\texttt{cardset[1]} &     0            &    0            & 0               & 0               &  0 & 0  & 0  &  0 &  0 & 0  &  1  & 0    & 0    &  0    & 8   \\
\texttt{cardset[2]} &     1            &    0            &  0              &  0              &  0 & 0  &  0 & 0  &  0 &  0 & 0   &  0   &  0   &  1    & 8193           \\
\texttt{cardset[3]} &     0            &     0           &  0              &  0              &   0 &   0&  0 &  0 &   0& 0  & 1  &  1   &  1   & 0     & 14      \\
\hline
\end{tabular}
\label{tab:CardsetBitfield}
\end{center}
\end{small}
\end{table}

\begin{table}[htb]
\captionsetup{position=top}
\caption[Valueset field]{The 'valueset' integer corresponding to the cardset bitfield example of \textbf{Table~\ref{tab:CardsetBitfield}}.
Every pair of bits denotes how many occurances of that value are in the hand.
In this case, there is one A, one Q, one J, three fours, a three, and a two.
The actual integer value is 72351957.}
\begin{small}
\begin{center}
\begin{tabular}{|r|c|c|c|c|c|c|c|c|c|c|c|c|c|c|c|c|}
\hline
Bits    &  27-26 & 25-24 & 23-22 & 21-20 & 19-18 & 17-16 & 15-14 & 13-12 & 11-10 & 9-8 & 7-6 & 5-4 & 3-2 & 1-0                       \\ \hline
Card    &    A    &     K & Q     & J     & 10    &     9 &     8 & 7     &     6 &  5  &   4 &   3 & 2   & A                          \\ \hline
Value  &    1    &0      & 1     &  1    &  0    & 0     &  0    &  0    &  0    &0    &  3  & 1   & 1   & 1     \\
\hline
\end{tabular}
\label{tab:ValuesetInteger}
\end{center}
\end{small}
\end{table}

The 'cardset' bitfield can efficiently determine if a certain card is contained in a hand.
The 'valueset' bitfield can efficienctly determine how many cards of a certain value are contained in the hand.

\subsection{Strength Calculation}
\label{sec:StrengthCalculation}

The first step of the strength calculation involves detecting what hand has been made.
Detecting straights and flushes take advantage of the 'cardset' representation.
Detecting flushes and matches (pair, two-pairs, full house, etc.) is done incrementally, as each card is added to the hand.

Straight flushes can be detected by shifting and ANDing each of the 'cardset' bitfields.
Straights can be detected by looking for straight flushes within a combined bitfield that is the bitwise OR of the four 'cardset' bitfields.
To detect flushes, any time a card is added to a hand, a corresponding counter is incremented for that suit.
Any counter exceeding five indicates that the hand contains a flush.
%by counting the number of 1s in each cardset bitfield (excluding the Ace at bit 0 so that it is not double-counted).
%Card matches (pairs, two-pairs, three-of-a-kind, full house, four-of-a-kind) are also detected incrementally.
Each hand stores the top two pairs that it contains, and the list is updated whenever another pair occurs.
If a card with a value that has already paired is added to a hand, the hand is promoted to a three-of-a-kind, or to a full house if another pair already exists.
The same applies to four-of-a-kind.

\subsection{Tiebreaking}
\label{sec:Tiebreaking}
Two hands that have the same hand

\subsection{Iteration}
\label{sec:Iteration}


\clearpage
