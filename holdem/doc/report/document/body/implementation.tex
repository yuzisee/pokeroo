
\newcommand{\xs}{$\spadesuit$}
\newcommand{\xh}{$\heartsuit$}
\newcommand{\xc}{$\clubsuit$}
\newcommand{\xd}{$\diamondsuit$}



\chapter{Poker Hand Evaluation}
\label{sec:HandEvaluation}

The core of the decision engine contains a mechanism for determining whether one set of cards is stronger than another.
%A hand evaluator is required in order to determine whether a set of cards is stronger than another.
This section presents a data structure for representation of a set of cards, as well as an efficient algorithm for computation of card strength for comparison.
The \texttt{CommunityCallStats} class implements the data structures discussed in Section~\ref{sec:DataStructures}.
The \texttt{DealRemainder} class implements the iteration algorithms discussed in Section~\ref{sec:IncompleteHands}.

%\section{Implementation}
%\label{sec:HandEvaluationImplementation}

%In order to generate probability distributions to describe $Pr_\mathrm{*} \left( f_\mathrm{bet} | H \right)$ it is necessary to implement a ... evaluator.

%The ... evaluator uses Holdem-specific assumptions to improve efficiency.

%For the purposes of implementation it is safe to assume that strengths will only be compared between seven-card hands.

Hand evaluation exists at two levels.
Firstly, it must be possible to compute which seven-card hand would win among a set of seven-card hands.
This requires an implicit computation of the best five-card poker hand that can be formed within each seven-card hand.
Secondly, it will be necessary to evaluate the strength of a set of incomplete hands, that is, those with fewer than seven cards.
This is determined by the distribution of potential seven-card hands that can be formed.


\section{Poker Hands}
\label{sec:PokerHands}


From strongest to weakest the standard poker hands are:

\begin{itemize}
\singlespacing
\item Straight Flush: Five consecutive card values, all with the same suit
\item Four-of-a-kind: Four matching cards
\item Full house: Three-of-a-kind and a different pair
\item Flush: Five cards with the same suit
\item Straight: Five consecutive card values, any suit
\item Three-of-a-kind: Three matching cards
\item Two-pair: Two different pairs
\item One pair: Two matching cards
\end{itemize}
In pair-type poker hands (e.g. two-pair, full house, etc.) a higher valued match is stronger than a lower valued match.
In a full house the three-of-a-kind is more important than the pair.
For example, 555QQ is stronger than 444KK, because three fives is stronger than three fours.
Similarly, 99332 is stronger than 7766K: the highest pair, nines, is stronger than the opposing highest pair, sevens.
In non-pair-type poker hands, or when the matches are identical between both players, the values of the remaining cards are used to break ties.
The highest tiebreaking card wins.
Aces are always considered high except when used for the ace-low (5432A) straight, and suits have no associated strength.
%Since all poker hands are five card hands, the poker hands that don't use all five cards (e.g. four-of-a-kind, two-pair, etc.) use the remaining cards to break ties.
If it is still a tie, the next highest tiebreaking card is compared next, until all five cards have been used.
In the case that none of the listed poker hands can be made, all five cards are tiebreak cards -- this is commonly referred to as a 'high card' hand.
A ``split pot'' occurs if a tie cannot be broken with all five cards.

Holdem poker uses these standard five-card poker hands for strength comparison, however each player is only dealt two cards of their own (known as ``hole cards'').
Five more cards are dealt as ``community cards'' and are equally available to all players.
%players always have seven available cards from which they will form the best five-card poker hand.
When determining the winner, each player forms their own best five-card poker hand using any five of the seven total available cards.


\section{Betting Rounds}
\label{sec:BettingRounds}

The only complex decisions that need to be made take place during the betting rounds.
In holdem poker, each game round consists of four betting rounds:
\begin{enumerate}
\singlespacing
\item Pre-flop: After each player has received his/her two hole cards, and before any community cards are dealt.
\item Post-flop: After the first three community cards are dealt.
\item Turn: After the fourth community card is dealt.
\item River: After the final (fifth) community card is dealt.
\end{enumerate}

During each betting round, each player is given at least one chance to bet.
A player's bet must match (known as a call) or exceed (known as a raise) the highest bet so far during the betting round, otherwise, a player forfeits (known as a fold) his/her opportunity to win until new hole cards are dealt in the following game round.
A betting round continues until all players have called or folded.
Players that fold make no further bets until the next hole cards are dealt.

After all four betting rounds have completed, the ``showdown'' takes place.
The player that forms the best five-card hand wins the total sum of all bets made in all betting rounds.
When there is a tie for best five-card hand between more than one player, the winnings are split evenly between the tying hands.



\section{Data Structures}
\label{sec:DataStructures}

The core poker hand evaluator takes a seven cards as input, and produces a comparable card `strength' object as output.
The strength of any seven cards can be represented with two values: the poker hand made and the tiebreaking cards.

By storing a set of cards as bitfields, the strength information can be computed efficiently.
A set of cards is stored as an array of four 16-bit integers (the `cardset' array), and a single 32-bit integer (the `valueset' integer).

Each of the four 16-bit `cardset' integers represents an individual suit.
Each bit in each `cardset' integer determines whether that card is contained in the set.
Storing aces as both bit 0 and bit 13 improves computational efficiency during the strength calculation stage.
\textbf{Table~\ref{tab:CardsetBitfield}} provides an example of the `cardset' bitfield.

The `valueset' integer determines how many times each card is contained in the set.
Every two bits are used to store a number between 0 and 3.
Although this representation is unable to store hands with four-of-a-kind, it will be shown that the the valueset variable is not needed in four-of-a-kind situations.
\textbf{Table~\ref{tab:ValuesetInteger}} provides an example of the `cardset' bitfield.

\begin{table}[htb]
\captionsetup{position=top}
\caption[Cardset Bitfield]{Each integer in the cardset array represents a separate suit.
In this example, index 0, 1, 2, and 3 represent spades, hearts, clubs, and diamonds, respectively.
The set of cards represented by (3080, 8, 8193, 14) is Q\xs{} J\xs{} 3\xs{} 4\xh{} A\xc{} 4\xd{} 3\xd{} 2\xd{}.
Bits 14 and higher are unused.}
\begin{small}
\begin{center}
\begin{tabular}{|r|c|c|c|c|c|c|c|c|c|c|c|c|c|c|c|c|r|}
\hline
Bit                 &  \ordinalnum{13} & \ordinalnum{12} & \ordinalnum{11} & \ordinalnum{10} & \ordinalnum{9} & \ordinalnum{8} & \ordinalnum{7} & \ordinalnum{6} & \ordinalnum{5} & \ordinalnum{4} & \ordinalnum{3} & \ordinalnum{2} & \ordinalnum{1} & \ordinalnum{0} &                        \\ \hline
Card                &                A &               K & Q               & J               & 10             &              9 &              8 & 7              &              6 &  5             &              4 &              3 & 2              & A              &                         \\ \hline
                    &                  &                 &                 &                 &   &   &   &   &   &   &   &   &                         &                     & Integer                       \\ \hline
\texttt{cardset[0]} &     0            &        0        & 1               &  1              &  0 & 0  &  0 &  0 &  0 &0   &  1  & 0    & 0    & 0     & 3080   \\
\texttt{cardset[1]} &     0            &    0            & 0               & 0               &  0 & 0  & 0  &  0 &  0 & 0  &  1  & 0    & 0    &  0    & 8   \\
\texttt{cardset[2]} &     1            &    0            &  0              &  0              &  0 & 0  &  0 & 0  &  0 &  0 & 0   &  0   &  0   &  1    & 8193           \\
\texttt{cardset[3]} &     0            &     0           &  0              &  0              &   0 &   0&  0 &  0 &   0& 0  & 1  &  1   &  1   & 0     & 14      \\
\hline
\end{tabular}
\label{tab:CardsetBitfield}
\end{center}
\end{small}
\end{table}

\begin{table}[htb]
\captionsetup{position=top}
\caption[Valueset field]{The `valueset' integer corresponding to the cardset bitfield example of \textbf{Table~\ref{tab:CardsetBitfield}}.
Every pair of bits denotes how many occurrences of that value exist in the set of cards.
In this case, there is one A, one Q, one J, three fours, a three, and a two.
The actual integer value is 72351957.}
\begin{small}
\begin{center}
\begin{tabular}{|r|c|c|c|c|c|c|c|c|c|c|c|c|c|c|c|c|}
\hline
Bits    &  27-26 & 25-24 & 23-22 & 21-20 & 19-18 & 17-16 & 15-14 & 13-12 & 11-10 & 9-8 & 7-6 & 5-4 & 3-2 & 1-0                       \\ \hline
Card    &    A    &     K & Q     & J     & 10    &     9 &     8 & 7     &     6 &  5  &   4 &   3 & 2   & A                          \\ \hline
Value  &    1    &0      & 1     &  1    &  0    & 0     &  0    &  0    &  0    &0    &  3  & 1   & 1   & 1     \\
\hline
\end{tabular}
\label{tab:ValuesetInteger}
\end{center}
\end{small}
\end{table}

The `cardset' bitfield efficiently determines whether a certain card is contained in the hand.
The `valueset' bitfield efficiently determines how many cards of a certain value are contained in the hand.
Furthermore, the ordinality of the `valueset' bitfield is very similar to what is required for tiebreaking.

%For the purposes of strength calculation,

\subsection{Strength Calculation}
\label{sec:StrengthCalculation}

The first step of the strength calculation involves detecting what poker hand has been made.
Detection of straights, straight flushes, and four-of-a-kinds takes advantage of the `cardset' representation.
Detection of pairs, triples and flushes is performed incrementally, as each card is added to the set.
The strength of a hand is stored in one byte as outlined in \textbf{Table~\ref{tab:StrengthByte}}.

Straight flushes can be detected by shifting and ANDing each of the `cardset' bitfields.
Straights can be detected by looking for straight flushes within a combined bitfield that is the bitwise OR of the four `cardset' bitfields.
To detect flushes, any time a card is added to the set, a corresponding counter is incremented for that suit.
Any counter exceeding five indicates that the poker hand contains at least a flush.
%by counting the number of 1s in each cardset bitfield (excluding the Ace at bit 0 so that it is not double-counted).
%Card matches (pairs, two-pairs, three-of-a-kind, full house, four-of-a-kind) are also detected incrementally.
A list of the top two pairs is always stored, and the list is incrementally updated whenever another pair is formed.
If a card with a value that has already paired is added again to a set of cards, the hand is promoted to `three-of-a-kind', or to `full house' if another pair already exists.
The same applies to four-of-a-kind.


\subsection{Tiebreaking}
\label{sec:Tiebreaking}
Two hands that have the exact same strength (e.g. two-pair vs. the same two-pair) always require tiebreaking.
For a set of five cards where either a flush or no poker hand is made, the ordinality of the `valueset' bitfield directly determines the winner.
For a set of seven cards, the two cards with the least value need to be subtracted from the `valueset' bitfield.
This operation is sufficient for all hands up to and including three-of-a-kind: The smallest two cards that aren't used in a pair/triplet are subtracted from the `valueset' bitfield, and the remaining `valueset' represents the tiebreaker.
The tiebreak value of a straight or straight flush only depends on one of the cards from the straight, and the tiebreak value of a four-of-a-kind depends only on the highest remaining card.
Lastly, during a full-house, the tiebreak variable needs only to represent the contained three-of-a-kind and pair, with more significance placed on the three-of-a-kind.

\begin{table}[htb]
\captionsetup{position=top}
\caption[Hand Strength Byte]{The strength of a poker hand can be represented with a single unsigned byte.
%The ordinality of each number reflects the relative strength of its corresponding poker hand.
There are 13 possible pairs, three-of-a-kinds, and four-of-a-kinds: A,K,...,3,2.
There are $\binom{13}{2}$ possible two-pairs: AK,AQ,AJ,...,54,53,52,43,42,32.}
\begin{small}
\begin{center}
\begin{tabular}{|r|c|l|}
\hline
Poker Hand      & Strength & Tiebreaker                                   \\ \hline
Straight Flush  & 122      & High-card of straight                        \\%& Ace is high except in 5432A straight \\
Four-of-a-kind  & 121-109  & Highest non-four-of-a-kind card              \\
Full house      & 108      & Three-of-a-kind, then best pair              \\
Flush           & 107      & Valueset, lowest two cards removed           \\
Straight        & 106      & High-card of straight                        \\
Three-of-a-kind & 105-93   & Valueset, lowest two unpaired cards removed  \\
Two-pair        & 92-15    & Valueset, lowest two unpaired cards removed  \\
One Pair        & 2-14     & Valueset, lowest two unpaired cards removed  \\
Nothing         & 1        & Valueset, lowest two cards removed           \\
\hline
\end{tabular}
\label{tab:StrengthByte}
\end{center}
\end{small}
\end{table}

Organizing the strength and tiebreak fields in this way maps each poker hand strength to a set of integers, while preserving ordinality.
This representation allows for efficient sorting and searching of arbitrary groups of seven-card hands, leading to a mechanism for evaluating incomplete hands by iterating over all potential complete hands that can be obtained.
The combinations of all strength and tiebreak pairs are summarized in \textbf{Table~\ref{tab:StrengthByte}}.



\section{Incomplete Hands}
\label{sec:IncompleteHands}

The strength of an incomplete hand is represented not by a single value, but by a distribution of values.
This distribution can be computed by iterating over all possible seven-card hands that can be obtained. %all potential combinations of the cards remaining in the deck.
So far, hand strength evaluation has been based on the strength of a single seven-card hand.
However, in game situations, hand strength is always relative to the opponent's hand.
Thus, given the hole cards of a specific player, the decision engine calculates the joint distribution of community cards and opposing hole cards.
A player might also choose to better position him/herself for the next betting round, rather than to achieve the greatest outright showdown utility.
For this reason, it is often desirable to distinguish between the immediately upcoming community cards, and future community cards.
%to condition the distribution on the specific community cards that will be dealt between the current betting round
The calculation of such distributions entail a very large runtime complexity.

The most computationally expensive distributions to calculate are for the `Pre-flop' betting round.
At this point a player has only his/her two hole cards and must consider $\binom{50}{3}$ potential combinations of immediately upcoming flops, leading to $\binom{47}{2}$ potential seven-card hands per flop, and $\binom{45}{2}$ possible combinations of opposing hole cards.
This requires the computation of roughly twenty billion combinations, and considers only one opponent.
A series of optimizations and approximations must be made to make the problem more computationally feasible. %reduce the problem into a more computationally feasible one.

\subsection{Optimizations}
\label{sec:Optimizations}

Certain optimizations that take advantage of symmetry among suits are employed to reduce the breadth of iterations required to generate the complete distributions in question.
For example, consider the hole card pair of 7\xs{} 4\xs{}.
A na\"{i}ve approach would be to explore:
\begin{itemize}
\singlespacing
\item 7\xs{} 4\xs{} 2\xs{} ...
\item 7\xs{} 4\xs{} 2\xh{} ...
\item 7\xs{} 4\xs{} 2\xc{} ...
\item 7\xs{} 4\xs{} 2\xd{} ...
\item 7\xs{} 4\xs{} 3\xs{} ...
\item 7\xs{} 4\xs{} 3\xh{} ...
\item ...
\end{itemize}
There is, however, a symmetry between the set of outcomes starting with 7\xs{} 4\xs{} 2\xh{}, the set of outcomes starting with 7\xs{} 4\xs{} 2\xc{}, and set of outcomes starting with 7\xs{} 4\xs{} 2\xd{}
%along the above three exploration paths
, and it is unnecessary to explore more than one of the three paths.
In this particular example it would be sufficient to explore only the outcomes starting with 2\xh{} given triple weight, reducing the branching factor by three:
\begin{itemize}
\singlespacing
\item 7\xs{} 4\xs{} 2\xs{} ...
\item 7\xs{} 4\xs{} 2\xh{} ...
\item \sout{7\xs{} 4\xs{} 2\xc{}} ...
\item \sout{7\xs{} 4\xs{} 2\xd{}} ...
\item 7\xs{} 4\xs{} 3\xs{} ...
\item 7\xs{} 4\xs{} 3\xh{} ...
\item \sout{7\xs{} 4\xs{} 3\xc{}} ...
\item ...
\end{itemize}



In another example, consider the hole card pair of 5\xs{} 5\xh{}.
Notice that adding 2\xs{} and then K\xh{} is symmetric to adding K\xs{} and then 2\xh{}, and it is unnecessary to explore both paths:
\begin{itemize}
\singlespacing
\item ...
\item 5\xs{} 5\xh{} 2\xs{} 2\xh{} ...
\item \sout{5\xs{} 5\xh{} 2\xs{} 3\xh{}} ...
\item \sout{5\xs{} 5\xh{} 2\xs{} 4\xh{}} ...
\item ...
\item \sout{5\xs{} 5\xh{} 2\xs{} Q\xh{}} ...
\item \sout{5\xs{} 5\xh{} 2\xs{} K\xh{}} ...
\item \sout{5\xs{} 5\xh{} 2\xs{} A\xh{}} ...
\item 5\xs{} 5\xh{} 3\xs{} 2\xh{} ...
\item 5\xs{} 5\xh{} 3\xs{} 3\xh{} ...
\item \sout{5\xs{} 5\xh{} 3\xs{} 4\xh{}} ...
\item ...
\item 5\xs{} 5\xh{} K\xs{} 2\xh{} ...
\item 5\xs{} 5\xh{} K\xs{} 3\xh{} ...
\item 5\xs{} 5\xh{} K\xs{} 4\xh{} ...
\item ...
\end{itemize}
It is possible to prune each of these duplicate paths by implicitly sorting the suits in descending order of cardset integers.
Because the initial cards contained symmetric spades and hearts, you can safely prune all potential seven-card hands where the hearts cardset is greater than the spades cardset.
In this particular example, the exploration algorithm avoids exploring all paths where the hearts cardset integer exceeds the spades cardset integer.

%At this point, spades has a larger cardset integer than hearts unless a heart larger than 3\xh{} is added to the hand.
%Instead,
%Exploiting such symmetries further reduces the branching factor.

For the pre-flop case in particular, combining the optimizations described above improves computation speed by 400\%.

\subsection{Caching}
\label{sec:Caching}

Finally, since only 169 unique hole card pairs (accounting for symmetry among suits) exist, it is possible to pre-compute these 169 distributions and cache them on disk.
This way, only post-flop or simpler distributions require to be calculated in real-time, with a worst case runtime complexity on the order of one million hand evaluations ($\binom{47}{2}$ seven-card hands and $\binom{45}{2}$ possible opponents).
This runtime complexity is then improved even further once applying the symmetry optimizations described in section~\ref{sec:Optimizations}.

\subsection{Approximations}
\label{sec:Approximations}

One additional runtime complexity reduction is to assume that the probabilities of winning against each opponent are independent.
This is not the case in general: Two opponents cannot simultaneously hold any of the same hole cards.
%If one opponent has weak cards, since the next opponent will not simultaneously hold those cards the distribution of possible hands
That is, knowledge of the hole cards of one opponent affects the distribution of hole cards of another opponent.
However, when the number of opponents is small, this assumption should be relatively safe.


%\subsection{Unknown Opponents}
%\label{sec:UnknownOpponents}




%Three such distributions are computed.

%\clearpage
