
\clearpage

%\cite{eulerAngles2008}



\chapter{Situational Evaluation}
\label{sec:SituationalEvaluation}

When deciding on an action to take, a competent poker player must always take certain factors into account.
The combination of all of the following factors is a player's betting situation:
\begin{itemize}
\singlespacing
\item $\mathrm{Pr\{raised_x\}}$: Probability of being raised by an opponent, to $x$ chips
\item $\mathrm{Pr\{push_y\}}$: Probability of causing all opponents to fold after betting $x$ chips
\item $\mathrm{Pr\{win\}}$: Probability of winning in a showdown
\item $\mathrm{E[callers]}$: Number of opponents expected to call or raise your bet
\end{itemize}
Also considered is the probability of improving the betting situation for the next betting round, as well as best-case and worst-case $Pr_\mathrm{win}$ estimates.

The estimation of each of the above four values is accomplished using two distributions.
\begin{itemize}
\singlespacing
\item \texttt{CallStats}: The possible hands your opponent can have, compared to the hand the you have
\item \texttt{CommunityCallStats}: The possible hands any player can have, compared to each other
\end{itemize}
The \texttt{CallStats} distribution helps determine how your opponent should behave if he/she knows what hand you have.
The \texttt{CommunityCallStats} is used to measure the ''rank'' of any given hand with respect to the community cards that are available.

The following sections will explore the generation and characteristics of each of these distributions, and then describe the computation of $\mathrm{Pr\{raised_x\}, Pr\{push_x\}, Pr\{win\}, and E[callers]}$.


\section{CallStats Distribution}
\label{sec:CallStats}

The \texttt{CallStats} distribution is a description of all of the possible seven-card hands that a player can form, compared to each of the hole cards that an opponent could have.
For a given incomplete hand, the \texttt{CallStats} distribution is computed by first iterating over all possible combinations of two opposing hole cards.
Then, for each opposing hole card combination, iterate over all of the possible community card combinations.
Each community card combination completes a seven-card hand for both the player and the opponent.
The seven-card hands are compared to determine the winner, and the results for each hole card combination are accumulated to generate $\mathrm{Pr\{win\}}$ against each opposing pair of hole cards.
\texttt{CallStats} is the histogram of 

The \texttt{CallStats} distribution, therefore, can also be treated as a function: $\mathrm{CallStats} : My Hole Cards, 



\section{Number of Callers}
\label{sec:Callers}





%Glossary: Pre-flop, post-flop, Flop, Turn, River, Split Pot, Showdown, Betting Round, Game Round, Five-card Poker Hand, Seven-card hand

\clearpage
