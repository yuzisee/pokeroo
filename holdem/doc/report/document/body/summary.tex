


\chapter{Concluding Summary}
\label{sec:Summary}

Ring Games -- EU

f: bet fraction

v: payoff function (At it{\textquotesingle}s core: vwin = num\_callers {\texttimes} f, but this becomes more complex later) 

Earn $\left((1+v_{\mathit{win}}\left(f\right))^{\mathit{wins}}(1-v_{\mathit{loss}}\left(f\right))^{\mathit{losses}}\right)\cdot B$ after (wins+losses+ties = n) games.


\bigskip


\bigskip

After including splits,  $U_{\mathit{showdown}}\left(f\right)=\prod _{e\;\in \;\mathit{outcomes}}\left((1+v_{e}\left(f\right))^{\mathit{Pr}_{e}\left(f\right)}\right)$ 

(We can ignore B because it is constant of f)

e can be: win, lose, split with 1, split with 2, split with 3, etc.

Notice that Pre is a function of f, see {\textquotedblleft}Actions{\textquotedblright}


\bigskip

We assume approximate derivative monotonicity over f, and search for maximum expected utility (Quadratic optimization, Newton{\textquotesingle}s method, etc.)


\bigskip

As f approaches 1, we can also evaluate ICM or approximate it with linear utility.


\bigskip

No Limit -- EU

An opponent may fold, call, or (re)raise, depending on his/her current hand H.

\begin{equation*}
U_{\mathit{bet}}\left(f_{\mathit{bet}}\right)=\left(U_{\mathit{fold}}\right)^{\mathit{Pr}_{\mathit{fold}}\left(f_{\mathit{bet}}\right)}\left(U_{\mathit{showdown}}\left(f_{\mathit{bet}}\right)\right)^{\mathit{Pr}_{\mathit{call}}\left(f_{\mathit{bet}}\right)}\prod _{f_{\mathit{raised}}}\left(U_{\mathit{showdown}}\left(f_{\mathit{bet}}+f_{\mathit{raised}}\right)\right)^{\mathit{Pr}_{\mathit{raise}}\left(f_{\mathit{bet}},f_{\mathit{raise}}\right)}
\end{equation*}
The opponent is more likely to fold weak hands to large bets: eg. Prwin(f) decreases with f, however, Ufold {\textgreater} 1 and Prfold(f) increases, which should produce a distinct optimum.

Prcall = 1 -- Prraise -- Prfold

Then, Prraise and Prfold can be looked up from the distributions below:

num\_callers in Ushowdown can also be derived from Prfold.


\bigskip

Distributions

We can find  $\mathit{Pr}_{\text{*}}\left(f_{\mathit{bet}}\right)$ (resp. $U_{\mathit{bet}}\left(f_{\mathit{bet}}\right)$) by iterating over  $\mathit{Pr}_{\text{*}}\left(f_{\mathit{bet}}|H\right)$ (resp.  $U_{\mathit{bet}}\left(f_{\mathit{bet}}|H\right)$) 

Possible distributions that can be used to define  $\mathit{Pr}_{\text{*}}\left(f_{\mathit{bet}}|H\right)$ are:

1. Assume the opponent knows your hand, what is their Prwin for each hand they could have?

2. Assume the opponent does not know your hand, what is the Prwin for each hand?

3. Order the hands in \#2 to provide a dimensionless rank ranging from 0 to 1; i.e. How many hands have a higher Prwin than your opponent?


\bigskip

Multiple opponents

We can approximate multiple-opponent probabilities initially.

\begin{equation*}
\mathit{Pr}_{\text{win all}}=\prod _{\mathit{opponents}}\mathit{Pr}_{\text{win single}}
\end{equation*}
\begin{equation*}
\mathit{Pr}_{\text{all fold}}=\prod _{\mathit{opponents}}\mathit{Pr}_{\text{single fold}}
\end{equation*}
\begin{equation*}
\mathit{Pr}_{\text{reraised}}=1-\prod _{\mathit{opponents}}\left(1-\mathit{Pr}_{\text{reraised single}}\right)
\end{equation*}
etc.


\bigskip

Fold Equity

Bots know they are bots, so we account for the fact that make the same decision in the same circumstances, indefinitely.


\bigskip

Say, p\% chance to win is the best hand in n hands. That means, \  $\frac{1}{1-p}=n$

If you fold, you are expecting to improve your odds and win the bet made over n hands.

In n identical chances, the best chance to win will be $p=1-\frac{1}{n}$

This means, if you allow yourself to wait n hands in a table of N opponents, you will win $\mathit{Pr}_{\text{win all}}=p^{N}=\left(1-\frac{1}{n}\right)^{N}$ of the time.

So if faced with a bet, and you choose to fold against a bet of b, you are saying

\begin{equation*}
U_{\text{fold equity}}=-{\frac{n}{\mathit{freq}}}C+\mathit{Pr}_{\mathit{win}}b-\mathit{Pr}_{\mathit{lose}}b
\end{equation*}
(Assuming you are pot committed C chips, in this situation, with frequency freq)


\bigskip

Opponent Fold Equity

The opponent{\textquotesingle}s chance of folding also affects their chance of winning the showdown. As your  $\mathit{Pr}_{\text{single fold}}$ increases against an opponent, your $\mathit{Pr}_{\text{win single}}$ also decreases simultaneously. An opponent that is likely to fold a bad hand will have a distribution of stronger hands when reaching the showdown. This balance between  $\mathit{Pr}_{\text{single fold}}$ and  $\mathit{Pr}_{\text{win single}}$ allows the bot to model the tradeoff between overbets vs. calculated bluffs.


\bigskip

Community Statistics

How your Prwin is affected by possible future community cards can also be measured. A heuristic for implied odds can be obtained from the skew of \ Prwin over all possible future community card combinations. A heuristic for opponent confidence can be obtained from the variance of \ Prwin over all possible future community card combinations. Community statistics can be used to develop {\textquotedblleft}playing styles{\textquotedblright} which can be dynamically selected throughout a match.

\chapter{Future Recommendations}
\label{sec:FutureDirection}

Areas for future development that have begun include:
\begin{itemize}
\singlespace
\item Expose public API for use in higher level languages
\item Create `playing-styles' or `personalities' for bots
\item Track opponent `playing-styles' or `personalities'
\end{itemize}

 modularizing the code base and exposing public APIs for access in higher level languages such as Python's extension framework, or Java's JNI.

 that may be more or less effective against specific types of players. University of Alberta currently employs this tactic in their leading poker AIs~\cite{PolarisCoached2007}
 
 
