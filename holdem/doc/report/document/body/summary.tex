


\chapter{Concluding Summary}
\label{sec:Summary}

This report establishes a working architecture for a complete computer poker AI designed to play no-limit holdem ring games.
Section~\ref{sec:HandEvaluation} designs a mechanism for efficiently evaluating the strength of complete and incomplete seven-card poker hands.
Section~\ref{sec:Histograms} describes how the incomplete hand strength evaluation techniques can be employed to generate the three basic histograms that represent the betting situation facing a poker bot.
%\begin{enumerate}
%\singlespacing
%\item \texttt{CallStats}: Assuming the opponent knows your hand, what is their $Pr\{win\}$ for each hand they could have?
%\item \texttt{CommunityCallStats}: Assume the opponent does not know your hand, what is the $Pr\{win\}$ for each hand?
%\item \texttt{WinStats}: Order the hands in \#2 to provide a dimensionless rank ranging from 0 to 1; i.e. How many hands have a higher Prwin than your opponent?
%\end{enumerate}
Section~\ref{sec:Utility} defines the structure of the utility functions that will govern the poker bot's behaviour:
\[
U = \left( \left( 1 + v_{\mathrm{win}} \left( f \right) \right)^N_w \left( 1 - v_{\mathrm{loss}} \left( f \right) \right)^N_l \right) \cdot B
\]
The basic formula is extended in the \texttt{GainModel} class to handle various split-pot situations together with specific computation of $v_{\mathrm{loss}}$ and $v_{\mathrm{win}}$ (using the $\mathrm{E[callers}]$ estimation of Section~\ref{sec:Callers}), forming the basis of the showdown utility function.

Such utility functions for the basis of all of the poker bot's decision making.
If the expected utility can be reduced to a single function of $b$, the proposed bet size, standard 1-D unimodal extremum search algorithms can be employed to determine an optimal bet size.
Section~\ref{sec:StateModel} describes how $\mathrm{Pr\{push}\}$ (of Section~\ref{sec:Push}), $\mathrm{Pr\{raised}\}$ (of Section~\ref{sec:Raises}), and showdown utility (of Section~\ref{sec:Utility}, requiring $\mathrm{E[callers}]$ of Section~\ref{sec:Callers}) are combined to form the overall utility function.
Heuristics for handling very large bet sizes are covered in Sections~\ref{sec:GeomAlgb}~and~\ref{sec:statworse}, and the key entrypoints for the main betting decision algorithm are presented in Section~\ref{sec:CompleteImplementation}.


%The underlying goal of the bot 
%The overall utility 

%An opponent may fold, call, or (re)raise, depending on his/her current hand H.

%\begin{equation*}
%U_{\mathit{bet}}\left(f_{\mathit{bet}}\right)=\left(U_{\mathit{fold}}\right)^{\mathit{Pr}_{\mathit{fold}}\left(f_{\mathit{bet}}\right)}\left(U_{\mathit{showdown}}\left(f_{\mathit{bet}}\right)\right)^{\mathit{Pr}_{\mathit{call}}\left(f_{\mathit{bet}}\right)}\prod _{f_{\mathit{raised}}}\left(U_{\mathit{showdown}}\left(f_{\mathit{bet}}+f_{\mathit{raised}}\right)\right)^{\mathit{Pr}_{\mathit{raise}}\left(f_{\mathit{bet}},f_{\mathit{raise}}\right)}
%\end{equation*}
%The opponent is more likely to fold weak hands to large bets: eg. Prwin(f) decreases with f, however, Ufold {\textgreater} 1 and Prfold(f) increases, which should produce a distinct optimum.

%Prcall = 1 -- Prraise -- Prfold

%Then, Prraise and Prfold can be looked up from the distributions below:

%num\_callers in Ushowdown can also be derived from Prfold.


%\bigskip

%Distributions

%We can find  $\mathit{Pr}_{\text{*}}\left(f_{\mathit{bet}}\right)$ (resp. $U_{\mathit{bet}}\left(f_{\mathit{bet}}\right)$) by iterating over  $\mathit{Pr}_{\text{*}}\left(f_{\mathit{bet}}|H\right)$ (resp.  $U_{\mathit{bet}}\left(f_{\mathit{bet}}|H\right)$) 

%Possible distributions that can be used to define  $\mathit{Pr}_{\text{*}}\left(f_{\mathit{bet}}|H\right)$ are:


% \bigskip

% Multiple opponents

% We can approximate multiple-opponent probabilities initially.

% \begin{equation*}
% \mathit{Pr}_{\text{win all}}=\prod _{\mathit{opponents}}\mathit{Pr}_{\text{win single}}
% \end{equation*}
% \begin{equation*}
% \mathit{Pr}_{\text{all fold}}=\prod _{\mathit{opponents}}\mathit{Pr}_{\text{single fold}}
% \end{equation*}
% \begin{equation*}
% \mathit{Pr}_{\text{reraised}}=1-\prod _{\mathit{opponents}}\left(1-\mathit{Pr}_{\text{reraised single}}\right)
% \end{equation*}
% etc.


% \bigskip

% Fold Equity

% Bots know they are bots, so we account for the fact that make the same decision in the same circumstances, indefinitely.


% \bigskip

% Say, p\% chance to win is the best hand in n hands. That means, \  $\frac{1}{1-p}=n$

% If you fold, you are expecting to improve your odds and win the bet made over n hands.

% In n identical chances, the best chance to win will be $p=1-\frac{1}{n}$

% This means, if you allow yourself to wait n hands in a table of N opponents, you will win $\mathit{Pr}_{\text{win all}}=p^{N}=\left(1-\frac{1}{n}\right)^{N}$ of the time.

% So if faced with a bet, and you choose to fold against a bet of b, you are saying

% \begin{equation*}
% U_{\text{fold equity}}=-{\frac{n}{\mathit{freq}}}C+\mathit{Pr}_{\mathit{win}}b-\mathit{Pr}_{\mathit{lose}}b
% \end{equation*}
% (Assuming you are pot committed C chips, in this situation, with frequency freq)


% \bigskip

% Opponent Fold Equity

% The opponent's chance of folding also affects their chance of winning the showdown. As your  $\mathit{Pr}_{\text{single fold}}$ increases against an opponent, your $\mathit{Pr}_{\text{win single}}$ also decreases simultaneously. An opponent that is likely to fold a bad hand will have a distribution of stronger hands when reaching the showdown. This balance between  $\mathit{Pr}_{\text{single fold}}$ and  $\mathit{Pr}_{\text{win single}}$ allows the bot to model the tradeoff between overbets vs. calculated bluffs.


% \bigskip

% Community Statistics

% How your Prwin is affected by possible future community cards can also be measured. A heuristic for implied odds can be obtained from the skew of \ Prwin over all possible future community card combinations. A heuristic for opponent confidence can be obtained from the variance of \ Prwin over all possible future community card combinations. Community statistics can be used to develop {\textquotedblleft}playing styles{\textquotedblright} which can be dynamically selected throughout a match.


% We assume approximate derivative monotonicity over f, and search for maximum expected utility (Quadratic optimization, Newton's method, etc.)


\chapter{Future Recommendations}
\label{sec:FutureDirection}

Areas for future development that are currently underway include:
\begin{itemize}
\singlespace
\item Exposing public API for integration in higher level languages
\item Implementing poker bots that have unique `playing-styles' or `personalities'
\item Tracking opponent `playing-styles' or `personalities' to better predict behaviour
\end{itemize}

By modularizing the code base and exposing public APIs for access in higher level languages such as Python's extension framework, or Java's JNI, it becomes possible to hook the poker bots into existing tournament frameworks and other poker software.
This leads to the potential for competition against other poker AI implementations, and is a significant opportunity to learn more about the performance of the poker bot.
%The easiest way to acheive this is to compile the C++ source code into a C DLL/.so library, and then employ the higher level languages' native execution tools, linking at the C level.

Implementing a variety of `playing-styles' or `personalities' can also be a powerful tool for improving the poker bot's performance, especially against human players.
A specific `playing-style' that deviates from an otherwise `optimal' approach that may turn out to be more effective against specific types of opponents, and less effective against others.
By choosing between the available `personalities' in real-time with a metaheuristic, it may be possible to dynamically identify the play style that best exploits the opponent's own style of play.
The University of Alberta's leading currently employs this tactic in their leading poker AIs~\cite{PolarisCoached2007}.

Lastly, 
same weaknesses of a human player that computers might otherwise desire to avoid.

\chapter{Costs}
There were no material costs, supplies, computer costs, or any other expenses that occurred due to this project.
